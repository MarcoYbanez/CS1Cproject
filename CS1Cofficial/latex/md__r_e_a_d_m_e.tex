2D Render Application

Goals\+: 1) Provide “contact us” method with team name and logo

2) Display all graphic objects (i.\+e. shapes including text) in rendering window. The shape id will be displayed above each shape identifying it. The rendering area to display shapes must have minimum dimensions of 1000 pixels (horizontal) by 500 pixels (vertical). The coordinate system is defined such that the top left corner of the rendering area is located at point (0,0), the bottom right corner at point (1000,500).

3) Your program should read from a shape file that keeps track of all shapes currently being rendered by the 2D modeler. Shapes are identified by their type\+: line, polyline, polygon, rectangle, ellipse, text. Shapes have properties\+: shape dimensions, pen color, pen width, pen style, pen cap style, pen join style, brush color, brush shape. \mbox{\hyperlink{class_text}{Text}} has properties\+: shape dimensions, text string, text color, text alignment, text point size, text font family, text font style, text font weight. All shapes must also have a unique ID.

4) Your program should be able to move shapes, including text, being rendered. This is accomplished via a move shape form. All changes are visible in the rendering area. –administrator only

5) Your program should be able to add and removeshapes, including text, being rendered.\+This is accomplished via an add/remove shape form. All changes are visible in the rendering area. –administrator only

6) Produce a shape listing report sorted by shape id (at any time). All shape properties should be included in the report.

7) Produce a shape listing report of O\+N\+LY shapes with an area sorted by area (at any time). The shape type, id and area should be included in the report.

8) Produce a shape listing report of O\+N\+LY shapes with a perimeter sorted by perimeter (at any time). The shape type, id and perimeter should be included in the report.

9) Save all changes between executions 